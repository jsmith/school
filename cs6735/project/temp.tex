\subsection{Motivations}
There exists several compelling reasons for studying generative modeling, especially GANs \citep{2017arXiv170100160G}. Generative models are extremely useful when your goal is understand the underlying data generating distribution. The ability to learn the joint probability distribution of high dimensional data is relevant to both applied math and engineering \citep{2017arXiv170100160G}.

Generative models are applicable to reinforcement learning, particularly model-based learning algorithms \citep{2017arXiv170100160G}. Moreover, there exits several techniques for applying GAN to semi-supervised learning problems which have proved themselves as reputable techniques. These two applications are expanded upon in section \ref{sec:applications}

Many applications require multi-modal outputs. For example, predicting the next action of a self-driving car. Within this context, it is important the model not be trained by traditional methods, such as using mean squared error to minimize the distance between the expected and predicted actions. These models cannot be used in situations where there is more than one appropriate prediction. Generative modeling, especially GAN, allow machine learning to function the scenarios where one input corresponds to multiple acceptable outputs \citep{2017arXiv170100160G}.

Tasks which involve image generation, modification or translation are all highly applicable to generative models. Although GANs may be used to generate any type of data, images remain the most commonly used source of training data. Research conducted within the past three years demonstrates the capabilities of GANs to perform these tasks \citep{2017arXiv170100160G, 2016arXiv161102200T, 2014arXiv1411.1784M}. For example, GANs have recently been shown to be capable of translating aerial photographs into maps and sketches into realistic images \citep{2017arXiv170100160G}.
